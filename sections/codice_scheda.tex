\section{Software Microcontrollore}
Il codice per la programmazione del microcontrollore si sviluppa dall'applicazione di base LoRaWAN\_End\_Node che fornisce
le funzioni necessarie per l'inizializzazione dei componenti nececessari e la comunicazione del microcontrollore attraverso il protocollo LoRa.
\\\\Per poter leggere i dati che si intendono acquisire dai sensori si è reso necessario organizzare e riformulare parte di codice per l'inizializzazione e il funzionamento del sensore
di accelerazione e il giroscopio. Nei prossimi paragrafi vengono rispettivamente presentate la strategia di collegamento dei sensori selezionati, la lettura dei dati, 
le modifiche apportate al codice per poter comunicare correttamente con il cloud, e il meccanismo grazie al quale si è reso possibile iniviare e ricevere i dati dal cloud.

\subsection{Gestione/lettura sensori}
% BREVE SCALETTA:
% 1) Descrizione breve del collegamento hardware e del bus I2C

% 2) Descrizione delle funzioni per l'inizializzazione e l' acquisizione dei sensori dall'almbiente esterno
% -Cenno a LSM6DSO_USER_Init() e la sua posizione nel codice
% -Cenno LSM6DSO_USER_Acc_GetAxes() e LSM6DSO_USER_Gyro_GetAxes()

% 3) Descrizione della funzione EnvSensorRead() e della struttura dati sensor_data per la lettura del sensore

\subsection{LoRaWAN}
% Breve descrizione del protocollo di rete
% -LoRaWAN_Init() per l'aggiornamento delle variabili
% -Cenno al DutyCicle per il tempo di lettura delle variabili? --> 3-4s ottimale
  
  \subsubsection{Uplink scheda}
  % Descrizione della funzione SendTxData
  % -Riferimento a variabile roll_state per decidere che dati inviare
  % -Strategia operativa per inserire i dati di sensor_data nel buffer


  \subsubsection{Downlink scheda}
  % Descrizione della funzione OnRxData:
  % -Riferimento a Variabile roll_state
  % -Lettura del buffer
  % -Accensione del led sulla scheda e cambiamento di stato



