\section{Obiettivo Progetto}
Si vuole realizzare un sistema nel quale il microcontrollore comunichi in modo bilaterale con un client mqtt tramite rete LoRa.
\\Il sistema genererà un numero casuale, basato sul risultato del lancio di $n$ dadi. Il sistema gestirà la quantità dei dadi e il loro lancio. Il client provvederà un'interfaccia grafica interattiva, in cui sarà possibile visualizzare la quantità dei dadi, il risultato del loro lancio
e cambiare lo stato del sistema in tempo reale, con dei pulsanti intuitivi.
\\Sono previste due modalità di funzionamento:
\begin{itemize}
  \item una che si occuperà di gestire la quantità dei dadi: bisognerà inclinare verso destra o sinistra per incrementare o decrementare la quantità
  \item una che si occuperà del lancio di essi: per lanciare i dadi, bisognerà scuotere la scheda
\end{itemize}
Il uC leggerà la modalità dal cloud, e in base ad essa invierà dati diversi.
\\Il client riceverà i dati e in base alla modalità li processerà in modo diverso. Quando si premerà il tasto per cambiare modalità, il client invierà al cloud il valore della nuova modalità scelta.





