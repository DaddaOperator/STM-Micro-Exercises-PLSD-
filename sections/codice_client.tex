\section{Client}
L'applicazione lato client viene sviluppata con il linguaggio di programmazione python, utilizzando l'insieme di moduli Pygame 
per la realizzazione dell'interfaccia grafica. (Ulteriori informazioni su pygame \href{https://www.pygame.org/}{https://www.pygame.org/})
\\\\Nel codice viene implementata la logica per garantire le funzionalità dell'applicazione, come l'interfaccia grafica, l'acquisizione 
dei dati dei sensori derivanti dal cloud, il cambiamento di stato e altre.\\\\
Lo stato corrente viene mantenuto da specifiche variabili di stato, che vengono modificate nel momento in cui si verificano eventi gestiti dai moduli pygame
(interazione del mouse con un pulsante presente sullo schermo). Inoltre, vengono mantenute ulteriori informazioni sui dati dell'applicazione in precise variabili:

\begin{verbatim}
roll_state = False      # variabile di stato corrente
dice_quantity = 1       # numero di dadi da lanciare
dice_state = []         # stato dei dadi usciti: es [6, 3, 1]
dice_generate = False   # lancia la funzione di generaz dadi
result = 0              # risultato finale
\end{verbatim}

\subsection{Gestione Uplink}


\subsection{Gestione Downlink}


\subsection{Interfaccia grafica}
L'interfaccia grafica viene implementata utilizzando i moduli per la generazione di elementi grafici messi a disposizione da pygame.
La scelta stilistica di utilizzare pygame è giustificata dalla necessità di rendere l'esperienza dell'utente più intuitiva possibile.
L'interfaccia grafica è realizzata in modo dinamico, per fare in modo che al cambiamento di stato del sistema, si modifichi in tempo reale.
\\\\Entrando nei dettagli della struttura del codice, l'interfaccia viene aggiornata con un massimo di 60 fotogrammi al secondo dalla funzione:
\begin{verbatim}
clock.tick(60)
\end{verbatim}
Nell'interfaccia grafica non vengono volutamente riportati i dettagli che riguardano la comunicazione tra il cloud e il client, ma solamente le funzionalità
indispensabili per una buona esperienza utente. 


















